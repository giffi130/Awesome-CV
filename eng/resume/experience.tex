\cvsection{Work Experience}
\begin{cvskills}
  \cvskill
    {Huray Positive Inc.} 
    {Apr. 2012 - Present} 
\end{cvskills}

\begin{cvexpentries}
  \cvexpentry
    {Python, Tensorflow, Numpy, PyDicom, OpenCV, MySQL}
    {Developing Brain CT Diagnosis support system based on machine learning}
    {Aug. 2016 - Developing}
    {}
    {
        \begin{cvitems}
            \item {진단 지원 Web 어플리케이션 설계}
            \item {병변 부위 Annotation Tool 설계 및 개발}
            \item {뇌 CT 전처리 알고리즘 개발}
            \item {뇌의 좌, 우 균형 분석용 뇌 중앙선 탐지 알고리즘 개발}
            \item {Deep Learning 학습 데이터 정제 및 분류}
            \item {Numpy를 활용한 이미지 픽셀 데이터 처리 성능 최적화}
            \item {진단 지원용 CNN Model 개발(뇌 내 출혈군, 지주막하 출혈군, 정상군)}
        \end{cvitems}
    }
\end{cvexpentries}

\begin{cvexpentries}
  \cvexpentry
    {Swift, CoreData, Alamofire, Bluetooth, Healthkit, Fabric(Crashlytics, Fastlane)}
    {임신성 당뇨 관리 iOS 어플리케이션 개발}
    {Dec. 2015 - On service}
    {}
    {
        \begin{cvitems}
            \item {어플리케이션 기획 참여, 전체 구조 설계 및 개발}
            \item {iOS Healthkit 활동량 분석 기능 개발}
            \item {\textbf{App URL.} \href{https://goo.gl/CHkNU5}{https://goo.gl/CHkNU5}}
            \item {\textbf{소개 페이지.} \href{https://momssens.com}{https://momssens.com}}
        \end{cvitems}
    }
\end{cvexpentries}

\begin{cvexpentries}
  \cvexpentry
    {Swift, CoreData, Alamofire, Bluetooth, Fitbit API, Fabric(Crashlytics, Fastlane)}
    {개인용 만성 질환 관리 iOS 어플리케이션 개발}
    {Jun. 2015 - On service}
    {}
    {
        \begin{cvitems}
            \item {어플리케이션 전체 구조 설계 및 개발}
            \item {BLE 혈당 측정기 연동 모듈 개발}
            \item {자체 개발 활동량계, Fitbit 활동량계 연동 기능 개발}
            \item {\textbf{App URL.} \href{https://goo.gl/itXLPq}{https://goo.gl/itXLPq}}
            \item {\textbf{소개 페이지.} \href{http://healthswitch.co.kr/}{http://healthswitch.co.kr}}
        \end{cvitems}
    }
\end{cvexpentries}

\begin{cvexpentries}
  \cvexpentry
    {Android, OpenCV, Fabric(Crashlytics), Nginx, Ruby on Rails, HTML, CSS, jQuery, MySQL}
    {유전자 증폭 시약 진단 지원 플랫폼 개발}
    {Sep. 2014 - Developing}
    {}
    {
        \begin{cvitems}
            \item {진단 지원 분석(Web) 및 수집(Android) 어플리케이션 설계 및 개발}
            \item {OpenCV를 이용한 촬영영역 인식 기술 개발}
            \item {HSV 색 공간을 이용한 시약 영역 탐지 알고리즘 개발}
            \item {MySQL 내 이미지 픽셀 데이터 처리 성능 최적화}
            \item {\textbf{특허.} 1020150142123 - 시약 키트의 이미지를 통한 감염병 진단 방법 및 장치}
        \end{cvitems}
    }
\end{cvexpentries}

\begin{cvexpentries}
  \cvexpentry
    {Ruby on Rails, Bootstrap, jQuery, POCT Protocol, HL7 Protocol, MySQL, Nginx}
    {웹 기반 병원용 혈당계 데이터 관리 시스템 개발}
    {Jan. 2014 - Jan. 2016}
    {}
    {
        \begin{cvitems}
            \item {시스템 전체 구조 설계 및 개발}
            \item {혈당계 연동 POCT Protocol 적용 및 Ruby 소켓 서버 개발}
            \item {EMR 연동 HL7 Protocol 적용}
            \item {EMR Database 사용자 정의 쿼리 생성 기능 개발}
            \item {\textbf{인증.} IHE HL7 Connectathon}
        \end{cvitems}
    }
\end{cvexpentries}

\begin{cvexpentries}
  \cvexpentry
    {Android, Ruby on Rails, Bootstrap, jQuery, Nginx, MySQL}
    {안드로이드 구글 글래스 DICOM 뷰어 어플리케이션 개발}
    {Jun. 2014 - Apr. 2015}
    {}
    {
        \begin{cvitems}
            \item {DICOM 관리 Web 어플리케이션 및 DICOM 뷰어 Android 어플리케이션 전체 설계 및 개발}
            \item {미국 Massachusetts General Hospital 영상의학과 협업: 의료 자문}
            \item {구글 글래스-서버간 통신 기능 개발}
            \item {이미지 종류 별 DICOM Preset 변경 기능 개발}
        \end{cvitems}
    }
\end{cvexpentries}

\begin{cvexpentries}
  \cvexpentry
    {Objective-C, CoreData, AFNetworking}
    {iOS 웹툰 어플리케이션 개발}
    {Mar. 2013 - Mar. 2014}
    {}
    {
        \begin{cvitems}
            \item {iOS 어플리케이션 전체 구조 설계 및 개발}            
            \item {중국 Shanda Games 협업}
        \end{cvitems}
    }
\end{cvexpentries}

\begin{cvexpentries}
  \cvexpentry
    {Ruby on Rails, Bootstrap, jQuery, AB1 Parser, MySQL, Nginx}
    {DNA 파일(AB1) 분석 웹 어플리케이션 개발}
    {Dec. 2012 - Jun. 2013}
    {}
    {
        \begin{cvitems}
            \item {파일 내 ATGC 데이터 분석 및 그래프 구현}
            \item {다수의 데이터 렌더링 성능 최적화}
            \item {돌연변이 탐지 알고리즘 개발}
        \end{cvitems}
    }
\end{cvexpentries}

\begin{cvexpentries}
  \cvexpentry
    {Ruby on Rails, MySQL, Nginx}
    {스타 웹진 모바일 어플리케이션 개발}
    {Jun. 2012 - Mar. 2013}
    {}
    {
        \begin{cvitems}
            \item {유명인사의 팬 SNS 서비스용 API 개발}
        \end{cvitems}
    }
\end{cvexpentries}

\begin{cvexpentries}
  \cvexpentry
    {Objective-C, Coredata, AFNetworking}
    {만성질환 자가 관리 iOS 어플리케이션 개발}
    {Jun. 2012 - Dec. 2013}
    {}
    {
        \begin{cvitems}
            \item {당뇨, 고혈압, 비만 관리 어플리케이션 구조 설계 및 개발}
        \end{cvitems}
    }
\end{cvexpentries}

\begin{cvexpentries}
  \cvexpentry
    {Android, Cocos2D, Java Socket}
    {재활 환자를 위한 게임 개발}
    {Jan. 2012 - Jun. 2012}
    {}
    {
        \begin{cvitems}
            \item {환자의 재활치료 효과 향상을 위한 게임 개발}
            \item {OS(Embeded 안드로이드)와 Socket 통신 기능 구현: 환자 움직임 값 수신}
            \item {Cocos 2D 엔진 활용으로 X, Y, Z 축용 게임 개발}
        \end{cvitems}
    }
\end{cvexpentries}

% \begin{cventries}
%   \cventry
%     {Software Engineer}
%     {Huray Positive Inc.}
%     {서울}
%     {2012. 4 - 현재}
%     {
%       \begin{cvitems}
%         \item {뇌 CT 진단 지원 시스템 개발}
%             \begin{cvsubentries}
%                 \cvsubentry{}{뇌 CT 진단 지원 시스템}{2016. 8 - 현재}{}
%                 \cvsubentry{}{두부 CT에 기계학습을 적용하여 병변의 종류와 위치를 찾아내어 의료진의 진단을 지원하는 시스템}{}{}
%                 \cvsubentry{}{기술. Python, Tensorflow, Numpy, OpenCV, MySQL}{}{}
%                 \cvsubentry{}{역할. 데이터 분석 및 시스템 개발}{}{}
%             \end{cvsubentries}
%         \item {유전자 증폭 시약 진단 지원 시스템 개발}
%             \begin{cvsubentries}
%                 \cvsubentry{}{유전자 증폭 시약 키트를 스마트폰으로 촬영하여 시약의 색상으로 양성/음성을 판단하는 시스템}{2014. 9 - 현재}{}
%                 \cvsubentry{}{기술. Android, Ruby on Rails, OpenCV, Fabric(Crashlytics)}{}{}
%                 \cvsubentry{}{역할. Android 개발, Ruby on Rails 개발}{}{}
%                 \cvsubentry{}{특허. 1020150142123 - 시약 키트의 이미지를 통한 감염병 진단 방법 및 장치}{}{}
%             \end{cvsubentries}
%         \item {개인용 만성 질환 관리 모바일 어플리케이션 개발}
%             \begin{cvsubentries}
%                 \cvsubentry{}{당뇨 환자를 위한 관리 어플리케이션}{2015. 6 - 현재}{}
%                 \cvsubentry{}{기술. Swift, CoreData, Alamofire, Fabric(Crashlytics, Fastlane)}{}{}
%                 \cvsubentry{}{역할. iOS 개발}{}{}
%                 \cvsubentry{}{App URL. \href{https://goo.gl/itXLPq}{https://goo.gl/itXLPq}}{}{}
%                 \cvsubentry{}{소개 페이지. \href{http://healthswitch.co.kr/}{http://healthswitch.co.kr/}}{}{}                
%             \end{cvsubentries}
%         \item {임신성 당뇨 관리 모바일 어플리케이션 개발}
%             \begin{cvsubentries}
%                 \cvsubentry{}{임신성 당뇨 환자를 위한 관리 어플리케이션}{2015. 12 - 현재}{}
%                 \cvsubentry{}{기술. Swift, CoreData, Alamofire, Fabric(Crashlytics, Fastlane)}{}{}
%                 \cvsubentry{}{역할. iOS 개발}{}{}
%                 \cvsubentry{}{App URL. \href{https://goo.gl/CHkNU5}{https://goo.gl/CHkNU5}}{}{}
%                 \cvsubentry{}{소개 페이지. \href{https://momssens.com}{https://momssens.com}}{}{}
%             \end{cvsubentries}
%         \end{cvitems}
%     }
% \end{cventries}

% \begin{cventries}
%   \cventry
%     {}
%     {}
%     {}
%     {}
%     {
%           \begin{cvitems}
%           \item {병원용 혈당계 데이터 관리 시스템 개발}
%             \begin{cvsubentries}
%                 \cvsubentry{}{POCT 표준에 맞춰 병원용 측정 기기를 관리하고 HL7 표준으로 병원 EMR과 연계되어 기기 QC, 환자 데이터, 통계를 분석하는 관리 프로그램}{2014. 1 - 2016. 1}{}
%                 \cvsubentry{}{기술. Ruby on Rails, POCT Protocol, HL7 Protocol}{}{}
%                 \cvsubentry{}{역할. Backend \& Frontend 개발}{}{}
%                 \cvsubentry{}{인증. IHE HL7 Connectathon}{}{}
%             \end{cvsubentries}
%         \item {구글 글래스 DICOM 뷰어 어플리케이션 개발}
%             \begin{cvsubentries}
%                 \cvsubentry{}{미국 Massachusetts General Hospital 영상의학과와 협업하여, 의료진이 구글 글래스 이용하여 영상을 상시 확인이 가능하도록 지원하는 어플리케이션}{2014. 6 - 2015. 4}{}
%                 \cvsubentry{}{기술. Android}{}{}
%                 \cvsubentry{}{역할. Android for Google Glass 개발}{}{}
%             \end{cvsubentries}
%         \item {웹툰 플랫폼 개발}
%             \begin{cvsubentries}
%                 \cvsubentry{}{중국 Shanda Games와 협력하여 개발한 웹툰 플랫폼}{2013. 3 - 2014. 3}{}
%                 \cvsubentry{}{기술. Objective-C, CoreData, AFNetworking}{}{}
%                 \cvsubentry{}{역할. iOS 개발}{}{}
%             \end{cvsubentries}
%         \item {DNA 파일(AB1) 분석 시스템 개발}
%             \begin{cvsubentries}
%                 \cvsubentry{}{개인의 유전자 파일을 불러와서 분석자가 원하는 지점을 찾아 보여주고, 돌연변이 여부를 찾아주는 시스템 }{2012. 12 - 2013. 6}{}
%                 \cvsubentry{}{기술. Ruby on Rails}{}{}
%                 \cvsubentry{}{역할. Frontend \& Backend 개발}{}{}
%             \end{cvsubentries}
%         \item {스타 웹진 모바일 어플리케이션 개발}
%             \begin{cvsubentries}
%                 \cvsubentry{}{유명인사의 팬들을 위한 SNS 어플리케이션}{2012. 6 - 2013. 3}{}
%                 \cvsubentry{}{기술. Ruby on Rails}{}{}
%                 \cvsubentry{}{역할. 모바일-서버 간 API 개발}{}{}
%             \end{cvsubentries}
%         \item {개인용 혈당 관리 모바일 어플리케이션 개발}
%             \begin{cvsubentries}
%                 \cvsubentry{}{개인용 혈당, 혈압, 체중 관리 어플리케이션}{2012. 6 - 2013. 13}{}
%                 \cvsubentry{}{기술. Objective-C, CoreData, AFNetworking}{}{}
%                 \cvsubentry{}{역할. iOS 개발}{}{}
%             \end{cvsubentries}
%         \item {재활 환자를 위한 게임 개발}
%             \begin{cvsubentries}
%                 \cvsubentry{}{재활치료의 효과 향상을 위해 개발하였으며, Embeded Android 하드웨어용으로 개발된 게임}{2012. 1 - 2012. 6}{}
%                 \cvsubentry{}{기술. Android, Cocos2D, Java Socket}{}{}
%                 \cvsubentry{}{역할. iOS 개발}{}{}
%             \end{cvsubentries}
%       \end{cvitems}
%     }
% \end{cventries}
